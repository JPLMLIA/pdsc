\documentclass[twocolumn]{extarticle}
\usepackage{fullpage}
\usepackage{amssymb}
\usepackage{sidecap}
\usepackage{units}
\usepackage{amsmath}
\usepackage{graphicx}
\usepackage{subfig}
\usepackage{hyperref}
\usepackage[margin=0.95in]{geometry}
\usepackage[all]{hypcap}
\usepackage{units}
\usepackage{upgreek}
\usepackage{acronym}

\hypersetup{ %
  pdftitle={},
  pdfauthor={},
  pdfkeywords={},
  pdfborder=0 0 0,
  pdfpagemode=UseNone,
  colorlinks=true,
  linkcolor=black,
  citecolor=black,
  filecolor=black,
  urlcolor=black,
  pdfview=FitH}

\acrodef{pds}[PDS]{Planetary Data System}
\acrodef{pdsc}[PDSC]{PDS Coincidences}

\sidecaptionvpos{figure}{t}

\makeatletter
\renewcommand\@maketitle{%
\noindent\begin{minipage}{0.95\textwidth}
%\vskip 0.7em
\let\footnote\thanks 
{\LARGE \@title \par }
\vskip 0.5em
{\large \@author \par}
\end{minipage}
\hfill
\vskip 2.0em \par
}
\makeatother

\title{
  PDSC: Planetary Data System Coincidences
}
\author{Gary Doran, Kiri Wagstaff, Liang Yu, and Lukas Mandrake}
\date{\today}

\begin{document}
\maketitle

\section*{Overview}
The \ac{pds} contains a wealth of publicly available information collected by
spacecraft from all across the solar system. In addition to hosting this
information, the \ac{pds} also provides capabilities to search through data
products by spacecraft, instrument, or even image content. However, several
other types of queries are also useful for those requiring cross-instrument
comparison or the comparison of images taken of the same region at different
times. Furthermore, users might desire the capability to query sets of
overlapping images programmatically, without relying on manually using a web
interface.

\section*{Solution}
To support queries of the type described above, we have developed a Python
library called \ac{pdsc}. The library ingests cumulative index files of
spacecraft imaging the surface of Mars from orbit, and allows querying images
(1) coincident with a given latitude/longitude location to within some
specified radius, or (2) overlapping some other image. The library is designed
to be extensible to other instrument types, and it could readily be extended to
handle images of planetary bodies other than Mars. The library is designed to
be fast, responding to most queries in less than a second.

\section*{Description}

\subsection*{Ingestion}
The \ac{pdsc} library allows for quickly querying coincident observations by
constructing a set of special databases and index structures during the
ingestion process. The first database constructed is simply a SQL version of the
PDS cumulative index. The second set of index structures holds geometric
information used for coincidence queries.

The footprints of observations on the surface of Mars are reported differently
across instruments. Some instruments report four corners of the observation in
latitude and longitude coordinates. Others report a latitude/longitude
location at the center of the observation, along with a ``north azimuth'' (the
direction that the ground track makes with north. There are several challenges
to using this information for determining which points fall within the
observation footprint. First, there might be errors in geo-registration, so that
the observation location is not entirely accurate. We do not attempt to solve
this challenge. The second challenge is that it takes more than four corners to
specify the precise boundary of an observation, since there is ambiguity over
how to connect such corners over long distances. For example, for long tracks,
the footprint is not bounded by geodesic lines. However, they are also not
connected by ``rhumb'' lines of constant bearing, especially over polar regions.

Thus, accurately describing a footprint requires projecting a geodesic line of
flight on the surface of the body, then moving perpendicularly to the line of
flight to determine the location of cross-track pixels. This method can be used
to decompose the observation footprint into smaller polygonal segments, which
are sufficiently small as to allow assuming that their vertices are connected by
simple geodesic lines. For reasons described later, we use triangular segments
to ensure that the segments are \emph{convex} polygons, a property that
simplifies the querying process.

These triangular segments are placed into a SQL database that records the
observation to which they belong along with the vertices that define their
corners. One final data structure saved to disk is a \emph{ball tree}, which is
populated with the set of segment centers. The radius (maximum distance from a
segment vertex to its center) is also recorded along with the tree. The use for
the ball tree, which allows querying all segment centers that fall within some
radius of a given query point, is also described below.

\subsection*{Queries}

There are two major query types supported by \ac{pdsc}. The first type of query
finds all observations within some distance $\epsilon$ of a given point on the
surface of Mars. For now, consider the case where $\epsilon = 0$, so the query
point must be \emph{within} the observation. To avoid having to check point
inclusion within every single observation, a pre-filtering step supported by the
index structures is performed. If a point falls within an observation, then it
falls within \emph{some} polygonal segment into which the observation has been
decomposed. Thus, we can reduce the problem to finding point inclusion within
triangular segments. Furthermore, if a point falls within a triangular segment,
then it is within radius $r$ from the segment center, where $r$ is the largest
distance from the center of the triangle to any of its vertices.

We use the ball tree index to quickly find all segments whose centers are at
most radius $r$ from the query point. This operation takes $O(\log n)$ time,
where $n$ is the number of segments in the database. Thus, we can significantly
reduce query time from the $O(n)$ operations needed to exhaustively search all
segments. After this pre-filtering step, we are left with a much smaller set of
segments to exhaustively check for point inclusion.

To check for point inclusion, 

\section*{Novelty}
pass

\bibliographystyle{plain}
\bibliography{bibliography}

\end{document}
