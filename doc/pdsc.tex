\documentclass[twocolumn]{extarticle}
\usepackage{fullpage}
\usepackage{amssymb}
\usepackage{sidecap}
\usepackage{units}
\usepackage{amsmath}
\usepackage{graphicx}
\usepackage{subfig}
\usepackage{hyperref}
\usepackage[margin=0.95in]{geometry}
\usepackage[all]{hypcap}
\usepackage{units}
\usepackage{upgreek}
\usepackage{acronym}

\hypersetup{ %
  pdftitle={},
  pdfauthor={},
  pdfkeywords={},
  pdfborder=0 0 0,
  pdfpagemode=UseNone,
  colorlinks=true,
  linkcolor=black,
  citecolor=black,
  filecolor=black,
  urlcolor=black,
  pdfview=FitH}

\acrodef{pds}[PDS]{Planetary Data System}
\acrodef{pdsc}[PDSC]{PDS Coincidences}

\sidecaptionvpos{figure}{t}

\makeatletter
\renewcommand\@maketitle{%
\noindent\begin{minipage}{0.95\textwidth}
%\vskip 0.7em
\let\footnote\thanks 
{\LARGE \@title \par }
\vskip 0.5em
{\large \@author \par}
\end{minipage}
\hfill
\vskip 2.0em \par
}
\makeatother

\title{
  PDSC: Planetary Data System Coincidences
}
\author{Gary Doran, Kiri Wagstaff, Liang Yu, and Lukas Mandrake}
\date{\today}

\begin{document}
\maketitle

\section*{Overview}
The \ac{pds} contains a wealth of publicly available information collected by
spacecraft from all across the solar system. In addition to hosting this
information, the \ac{pds} also provides capabilities to search through data
products by spacecraft, instrument, or even image content. However, several
other types of queries are also useful for those requiring cross-instrument
comparison or the comparison of images taken of the same region at different
times. Furthermore, users might desire the capability to query sets of
overlapping images programmatically, without relying an manually using a web
interface.

\section*{Solution}
To support queries of the type described above, we have developed a Python
library called \ac{pdsc}. The library ingests cumulative index files of
spacecraft imaging the surface of Mars from orbit, and allows querying images
(1) coincident with a given latitude/longitude location or (2) overlapping some
other image. The library is designed to be extensible to other instrument types,
and it could readily be extended to handle images of planetary bodies other than
Mars.

\section*{Description}
pass

\section*{Novelty}
pass

\bibliographystyle{plain}
\bibliography{bibliography}

\end{document}
